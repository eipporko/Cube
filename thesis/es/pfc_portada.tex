%
% Portada.
%

% Nota: Ser�a m�s c�modo emplear el comando \maketitle que genera una portada de forma autom�tica, pero 
% no incluye toda la informaci�n que es necesario incluir en la memoria de un proyecto de fin de carrera
% de la Facultad de Inform�tica de A Coru�a.
%

\begin{titlepage}

	\begin{center}

		% Logotipo de la universidad.
		\includegraphics[width=6cm]{../figures/anagrama.png}
		\vspace{2cm}

		% Nombre de la facultad, de la universidad y del departamento en que se realiza el PFC.
		{\Large{\textbf{Facultad de Inform�tica}}}
		\\
		{\it \large{\textbf{Departamento de Electr�nica y Sistemas}}}
		\vspace{1cm}

		% Indicamos el nombre de la titulaci�n oficial que hemos cursado con tanto esfuerzo.
		{\large PROYECTO DE FIN DE CARRERA\\INGENIER�A T�CNICA EN INFORM�TICA DE SISTEMAS}
		\vspace{1cm}

		% T�tulo
		\textbf{\Large Visualizaci�n avanzada de nubes de puntos con OpenGL}
		\vspace{7cm}
	\end{center}

	\begin{flushright}
		\begin{tabular}{ll}
			% Nombre del alumno.
			\large{\textbf{Alumno:}}	&
			\large{Ant�nez Gonz�lez, David} \\

			% Nombre del director/tutor del proyecto.
			\large{\textbf{Director:}}	&
			\large{�lvarez Mures, Luis Omar} \\

			% Nombre del director/tutor del proyecto.
			\large{\textbf{Tutor y Director:}}	&
			\large{Padr�n Gonz�lez, Emilio Jos�} \\
			
			% Fecha.
			\large{\textbf{Fecha:}}	&
			\large{\today} \\
		\end{tabular}
	\end{flushright}

\end{titlepage}
