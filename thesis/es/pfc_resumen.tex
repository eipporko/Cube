%
% Resumen del proyecto de fin de carrera
%

\section*{Resumen:}

El render de nubes de puntos adquiri� un renovado inter�es en los �ltimos a�os con la popularizaci�n y proliferaci�n de nuevos sistemas de adquisici�n de datos de medio coste tipo LiDAR o de bajo basados en fotogrametr�a o c�maras infrarojas.

Estos dispositivos obtienen una nube de puntos 3D (posicion geom�trica), con informaci�n adicional arbitraria asociada, normalmente el color como m�nimo, pero tambi�n es posible otros datos como normales, temperatura, etc. Por las propias caracter�sticas que tienen el punto como primitiva gr�fica, respecto por ejemplo de los tradicionales pol�gonos (no tienen area, no tienen orientaci�n, no est�n conectados, etc.), el render o visualizaci�n de nubes de puntos presenta una serie de retos si queremos obtener una visualizaci�n realista y de calidad. 

Este proyecto se centra en la visualizaci�n avanzada de nubes de puntos, aplicando algunas de las mas novedosas t�cnicas de render y haciendo uso de las caracter�sticas avanzadas de la API gr�fica multiplataforma OpenGL y explotando el hardware de las tarjetas gr�ficas modernas. El resultado del proyecto ser� la implementaci�n de una herramienta multiplataforma para la visualizaci�n avanzada de nubes de puntos 3D.
