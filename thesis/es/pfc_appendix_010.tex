%
% Memoria econ�mica.
%

\chapter[Shaders]{Shaders.}

\section{Sized-Fixed Points}

\textbf{Vertex Shader}
\begin{lstlisting}[frame=single]
#version 400
uniform mat4 viewMatrix, projMatrix;
uniform mat3 normalMatrix;
uniform bool colorEnabled;

in  vec3 in_Position;
in  vec3 in_Color;
in  vec3 in_Normals;

out vec3 ex_Color;

void main(void) {
	gl_Position = projMatrix * viewMatrix * vec4(in_Position, 1.0);
	gl_PointSize = 2;

	vec3 color = vec3 (0.0, 0.0f, 0.0f);

	//Diffuse
	if (colorEnabled == true) {
		vec3 lightDirection = vec3(0.0,0.0,1.0f);
		float dotValue = max(dot(normalize(normalMatrix * in_Normals), lightDirection), 0.0);
		ex_Color = vec3(dotValue) + color; }
	else
		ex_Color = in_Color;
}
\end{lstlisting}

\textbf{Fragment Shader}
\begin{lstlisting}[frame=single]
#version 400
uniform float n; //Near parameter of the viewing frustum
uniform float f; //Far parameter of the viewing frustum

in  vec3 ex_Color;
out vec4 out_Color;

void main(void) {	
	out_Color = vec4(ex_Color,1.0);
}
\end{lstlisting}
\vfill

\section{Image-aligned Squares}
\textbf{Vertex Shader}
\begin{lstlisting}[frame=single]
#version 400
uniform mat4 viewMatrix, projMatrix;
uniform mat3 normalMatrix;
uniform int h; //Height of the viewport
uniform float n; //Near parameter of the viewing frustum
uniform float t; //Top parameter of the viewing frustum
uniform float b; //Bottom parameter of the viewing frustum
uniform float userRadiusFactor; //Splat's radii
uniform bool colorEnabled;
uniform bool automaticRadiusEnabled;

in float in_Radius;
in  vec3 in_Position;
in  vec3 in_Color;
in 	vec3 in_Normals;

out float ex_Radius;
out vec3 ex_Color;

vec4 ccPosition; //position in Camera Coordinates

void main(void) {
	if (automaticRadiusEnabled == true)
		ex_Radius = in_Radius * userRadiusFactor;
	else
		ex_Radius = userRadiusFactor;

	ccPosition = viewMatrix * vec4(in_Position, 1.0);
	gl_Position = projMatrix * ccPosition;
	gl_PointSize = 2 * ex_Radius * (n / ccPosition.z) * (h / (t-b));

	vec3 color = vec3 (0.0, 0.0f, 0.0f);

	//Diffuse
	if (colorEnabled == true) {
		vec3 lightDirection = vec3(0.0,0.0,1.0f);
		float dotValue = max(dot(normalize(normalMatrix * in_Normals), lightDirection), 0.0);
		ex_Color = vec3(dotValue) + color; }
	else
		ex_Color = in_Color;
}
\end{lstlisting}

\textbf{Fragment Shader}
\begin{lstlisting}[frame=single]
#version 400
uniform float n; //Near parameter of the viewing frustum
uniform float f; //Far parameter of the viewing frustum

in  vec3 ex_Color;
out vec4 out_Color;

void main(void)
{	
	out_Color = vec4(ex_Color,1.0);
}
\end{lstlisting}
\vfill


\section{Affinely Projected Point Sprites}
\textbf{Vertex Shader}
\begin{lstlisting}[frame=single]
#version 400
uniform mat4 viewMatrix, projMatrix;
uniform mat3 normalMatrix;
uniform int h; //Height of the viewport
uniform float n; //Near parameter of the viewing frustum
uniform float t; //Top parameter of the viewing frustum
uniform float b; //Bottom parameter of the viewing frustum
uniform float userRadiusFactor; //Splat's radii
uniform bool colorEnabled;
uniform bool automaticRadiusEnabled;

in float in_Radius;
in  vec3 in_Position;
in  vec3 in_Color;
in 	vec3 in_Normals;

out float ex_Radius;
out vec3 ex_Color;
out vec3 ex_Normals;
out float ex_Pz; //z in Camera Coordinates

vec4 ccPosition; //position in Camera Coordinates

void main(void) {
	ex_Normals = normalize(normalMatrix * in_Normals);

	if (abs(ex_Normals.z) <= 0.1)
		ex_Normals.z = 0.1;

	if (automaticRadiusEnabled == true)
		ex_Radius = in_Radius * userRadiusFactor;
	else
		ex_Radius = userRadiusFactor;

	ccPosition = viewMatrix * vec4(in_Position, 1.0);
	gl_Position = projMatrix * ccPosition;
	gl_PointSize = 2*ex_Radius * (n / ccPosition.z) * (h / (t-b));

	//BackFace Culling
	if (dot (ccPosition.xyz, ex_Normals) > 0)
		gl_Position.w = 0;

	vec3 color = vec3 (0.0, 0.0f, 0.0f);

	//Diffuse
	if (colorEnabled == true) {
		vec3 lightDirection = vec3(0.0,0.0,1.0f);
		float dotValue = max(dot(normalize(normalMatrix * in_Normals), lightDirection), 0.0);
		ex_Color = vec3(dotValue) + color; }
	else
		ex_Color = in_Color;

	ex_Pz = ccPosition.z;
}
\end{lstlisting}

\textbf{Fragment Shader}
\begin{lstlisting} [frame=single]
#version 400
uniform float n; //Near parameter of the viewing frustum
uniform float f; //Far parameter of the viewing frustum

in  vec3 ex_Color;
in 	vec3 ex_Normals;
in float ex_Pz;

out vec4 out_Color;

vec3 test;
float zBuffer;

void main(void) {
	test.x = gl_PointCoord.x - 0.5;
	test.y = gl_PointCoord.y - 0.5;
	test.z = -(ex_Normals.x/ex_Normals.z) * test.x - (ex_Normals.y/ex_Normals.z) * test.y;
	if (length(test) > 0.5)
		discard;

	zBuffer = (ex_Pz + test.z * 0.1);
	gl_FragDepth = ((1.0 / zBuffer) * ( (f * n) / (f - n) ) + ( f / (f - n) ));
	
	out_Color = vec4(ex_Color,1.0);
}
\end{lstlisting}
\vfill

\section{Perspective Correct Rasterization}

\textbf{Vertex Shader}
\begin{lstlisting} [frame=single]
#version 400
uniform mat4 viewMatrix, projMatrix;
uniform mat3 normalMatrix;
uniform int h; //Height of the viewport
uniform float n; //Near parameter of the viewing frustum
uniform float t; //Top parameter of the viewing frustum
uniform float b; //Bottom parameter of the viewing frustum
uniform float userRadiusFactor; //Splat's radii
uniform bool automaticRadiusEnabled;

in float in_Radius;
in  vec3 in_Position;
in  vec3 in_Color;
in 	vec3 in_Normals;

out vec3 ex_Color;
out float ex_Radius;

out vec4 ccPosition; //position in Camera Coordinates
out vec3 normals;



void main(void) {
	normals = normalize(normalMatrix * in_Normals);

	if (automaticRadiusEnabled == true)
		ex_Radius = in_Radius * userRadiusFactor;
	else
		ex_Radius = userRadiusFactor;

	ccPosition = viewMatrix * vec4(in_Position, 1.0);
	gl_Position = projMatrix * ccPosition;
	gl_PointSize = 2 * ex_Radius * (n / ccPosition.z) * (h / (t-b));

	//BackFace Culling
	if (dot (ccPosition.xyz, normals) > 0)
		gl_Position.w = 0;

	ex_Color = in_Color;
}
\end{lstlisting}

\textbf{Fragment Shader}
\begin{lstlisting} [frame=single]
#version 400
uniform mat4 viewMatrix;
uniform float n; //Near parameter of the viewing frustum
uniform float f; //Far parameter of the viewing frustum
uniform float t; //Top parameter of the viewing frustum
uniform float b; //Bottom parameter of the viewing frustum
uniform float r; //Right parameter of the viewing frustum
uniform float l; //Left parameter of the viewing frustum
uniform int h; 	 //Height of the viewport
uniform int w; 	 //Width of the viewport
uniform bool colorEnabled;
uniform int lightCount;
uniform vec3 lightPosition[16];
uniform vec3 lightColor[16];
uniform float lightIntensity[16];

in float ex_Radius;
in  vec3 ex_Color;
in 	vec3 ex_UxV;
in  vec3 normals;
in 	vec4 ccPosition;

out vec4 out_Color;

void main(void) {

	vec3 qn;
	qn.x = (gl_FragCoord.x ) *  ((r - l)/w ) - ( (r - l)/2.0 );
	qn.y = (gl_FragCoord.y ) *  ((b - t)/h ) - ( (b - t)/2.0 );
	qn.z = -n;

	float denom = dot (qn, normals);

	if (denom == 0.0)
		discard;

	float timef = dot (ccPosition.xyz, normals ) / denom;

	vec3 q = qn * timef;
	vec3 testq = q;
	
	vec3 dist = (q - ccPosition.xyz);

	if ((dist.x * dist.x) + (dist.y * dist.y) + (dist.z * dist.z) > pow(ex_Radius, 2))
		discard;

	gl_FragDepth = ((1.0 / q.z) * ( (f * n) / (f - n) ) + ( f / (f - n) ));

	vec3 color = ex_Color;
	if (colorEnabled == true)
		color = vec3(0,0,0);

	//Diffuse
	vec3 dotValue = vec3(0,0,0);
	for (int i = 0; i < lightCount; i++) {
		vec3 ccLightPosition = (viewMatrix * vec4(lightPosition[i], 1.0f)).xyz;
		vec3 lithToQ = normalize(ccLightPosition - testq);
		dotValue += vec3(max(dot(normals, lithToQ), 0.0)) * lightIntensity[i] * lightColor[i];
	}

	out_Color = vec4(dotValue + color, 1.0f);
}
\end{lstlisting}