% 
% Apendice de shaders
% 
\lstset{basicstyle=\footnotesize,identifierstyle=\ttfamily,
  morekeywords=[1]{
    attribute, const, uniform, varying,
    layout, centroid, flat, smooth,
    noperspective, break, continue, do,
    for, while, switch, case, default, if,
    else, in, out, inout, float, int, void,
    bool, true, false, invariant, discard,
    return, mat2, mat3, mat4, mat2x2, mat2x3,
    mat2x4, mat3x2, mat3x3, mat3x4, mat4x2,
    mat4x3, mat4x4, vec2, vec3, vec4, ivec2,
    ivec3, ivec4, bvec2, bvec3, bvec4, uint,
    uvec2, uvec3, uvec4, lowp, mediump, highp,
    precision, sampler1D, sampler2D, sampler3D,
    samplerCube, sampler1DShadow,
    sampler2DShadow, samplerCubeShadow,
    sampler1DArray, sampler2DArray,
    sampler1DArrayShadow, sampler2DArrayShadow,
    isampler1D, isampler2D, isampler3D,
    isamplerCube, isampler1DArray,
    isampler2DArray, usampler1D, usampler2D,
    usampler3D, usamplerCube, usampler1DArray,
    usampler2DArray, sampler2DRect,
    sampler2DRectShadow, isampler2DRect,
    usampler2DRect, samplerBuffer,
    isamplerBuffer, usamplerBuffer, sampler2DMS,
    isampler2DMS, usampler2DMS,
    sampler2DMSArray, isampler2DMSArray,
    usampler2DMSArray, struct
  },
  morekeywords=[2]{
    radians,degrees,sin,cos,tan,asin,acos,atan,
    atan,sinh,cosh,tanh,asinh,acosh,atanh,pow,
    exp,log,exp2,log2,sqrt,inversesqrt,abs,sign,
    floor,trunc,round,roundEven,ceil,fract,mod,modf,
    min,max,clamp,mix,step,smoothstep,isnan,isinf,
    floatBitsToInt,floatBitsToUint,intBitsToFloat,
    uintBitsToFloat,length,distance,dot,cross,
    normalize,faceforward,reflect,refract,
    matrixCompMult,outerProduct,transpose,
    determinant,inverse,lessThan,lessThanEqual,
    greaterThan,greaterThanEqual,equal,notEqual,
    any,all,not,textureSize,texture,textureProj,
    textureLod,textureOffset,texelFetch,
    texelFetchOffset,textureProjOffset,
    textureLodOffset,textureProjLod,
    textureProjLodOffset,textureGrad,
    textureGradOffset,textureProjGrad,
    textureProjGradOffset,texture1D,texture1DProj,
    texture1DProjLod,texture2D,texture2DProj,
    texture2DLod,texture2DProjLod,texture3D,
    texture3DProj,texture3DLod,texture3DProjLod,
    textureCube,textureCubeLod,shadow1D,shadow2D,
    shadow1DProj,shadow2DProj,shadow1DLod,
    shadow2DLod,shadow1DProjLod,shadow2DProjLod,
    dFdx,dFdy,fwidth,noise1,noise2,noise3,noise4,
    EmitVertex,EndPrimitive
  },
  morekeywords=[3]{
    gl_VertexID,gl_InstanceID,gl_Position,
    gl_PointSize,gl_ClipDistance,gl_PerVertex,
    gl_Layer,gl_ClipVertex,gl_FragCoord,
    gl_FrontFacing,gl_ClipDistance,gl_FragColor,
    gl_FragData,gl_MaxDrawBuffers,gl_FragDepth,
    gl_PointCoord,gl_PrimitiveID,
    gl_MaxVertexAttribs,gl_MaxVertexUniformComponents,
    gl_MaxVaryingFloats,gl_MaxVaryingComponents,
    gl_MaxVertexOutputComponents,
    gl_MaxGeometryInputComponents,
    gl_MaxGeometryOutputComponents,
    gl_MaxFragmentInputComponents,
    gl_MaxVertexTextureImageUnits,
    gl_MaxCombinedTextureImageUnits,
    gl_MaxTextureImageUnits,
    gl_MaxFragmentUniformComponents,
    gl_MaxDrawBuffers,gl_MaxClipDistances,
    gl_MaxGeometryTextureImageUnits,
    gl_MaxGeometryOutputVertices,
    gl_MaxGeometryOutputVertices,
    gl_MaxGeometryTotalOutputComponents,
    gl_MaxGeometryUniformComponents,
    gl_MaxGeometryVaryingComponents,gl_DepthRange,
    \#version,core
  }
}

\chapter[Shaders]{Shaders.}

\section{Sized-Fixed Points \label{sized-fixed}}

\textbf{Vertex Shader}
\begin{lstlisting}[frame=single]
  #version 400
  uniform mat4 viewMatrix, projMatrix;
  uniform mat3 normalMatrix;

  in  vec3 in_Position;
  in  vec3 in_Color;

  out vec3 ex_Color;

  void main(void) {
    gl_Position = projMatrix * viewMatrix * vec4(in_Position, 1.0);
    gl_PointSize = 2;

    vec3 color = vec3 (0.0, 0.0f, 0.0f);

    ex_Color = in_Color;
  }
\end{lstlisting}

\textbf{Fragment Shader}
\begin{lstlisting}[frame=single]
  #version 400
  in  vec3 ex_Color;
  out vec4 out_Color;

  void main(void) {
    out_Color = vec4(ex_Color,1.0);
  }
\end{lstlisting}


\section{Image-aligned Squares \label{image-aligned}}
\textbf{Vertex Shader}
\begin{lstlisting}[frame=single]
  #version 400
  uniform mat4 viewMatrix, projMatrix;
  uniform mat3 normalMatrix;
  uniform int h; //Height of the viewport
  uniform float n; //Near parameter of the viewing frustum
  uniform float t; //Top parameter of the viewing frustum
  uniform float b; //Bottom parameter of the viewing frustum

  in float in_Radius;
  in  vec3 in_Position;
  in  vec3 in_Color;

  out float ex_Radius;
  out vec3 ex_Color;

  vec4 ccPosition; //position in Camera Coordinates

  void main(void) {
    ex_Radius = in_Radius;

    ccPosition = viewMatrix * vec4(in_Position, 1.0);
    gl_Position = projMatrix * ccPosition;
    gl_PointSize = 2 * ex_Radius * (n / ccPosition.z) * (h / (t-b));

    vec3 color = vec3 (0.0, 0.0f, 0.0f);

    ex_Color = in_Color;
  }
\end{lstlisting}

\textbf{Fragment Shader}
\begin{lstlisting}[frame=single]
  #version 400
  uniform float n; //Near parameter of the viewing frustum
  uniform float f; //Far parameter of the viewing frustum

  in  vec3 ex_Color;
  out vec4 out_Color;

  void main(void) {
    out_Color = vec4(ex_Color,1.0);
  }
\end{lstlisting}


\section{Affinely Projected Point Sprites \label{affinely}}
\textbf{Vertex Shader}
\begin{lstlisting}[frame=single]
  #version 400
  uniform mat4 viewMatrix, projMatrix;
  uniform mat3 normalMatrix;
  uniform int h; //Height of the viewport
  uniform float n; //Near parameter of the viewing frustum
  uniform float t; //Top parameter of the viewing frustum
  uniform float b; //Bottom parameter of the viewing frustum

  in float in_Radius;
  in  vec3 in_Position;
  in  vec3 in_Color;
  in  vec3 in_Normals;

  out float ex_Radius;
  out vec3 ex_Color;
  out vec3 ex_Normals;
  out float ex_Pz; //z in Camera Coordinates

  vec4 ccPosition; //position in Camera Coordinates

  void main(void) {
    ex_Normals = normalize(normalMatrix * in_Normals);

    if (abs(ex_Normals.z) <= 0.1)
      ex_Normals.z = 0.1;

    ex_Radius = in_Radius;

    ccPosition = viewMatrix * vec4(in_Position, 1.0);
    gl_Position = projMatrix * ccPosition;
    gl_PointSize = 2*ex_Radius * (n / ccPosition.z) * (h / (t-b));

    //BackFace Culling
    if (dot (ccPosition.xyz, ex_Normals) > 0)
      gl_Position.w = 0;

    vec3 color = vec3 (0.0, 0.0f, 0.0f);

    ex_Color = in_Color;

    ex_Pz = ccPosition.z;
  }
\end{lstlisting}

\textbf{Fragment Shader}
\begin{lstlisting} [frame=single]
  #version 400
  uniform float n; //Near parameter of the viewing frustum
  uniform float f; //Far parameter of the viewing frustum

  in  vec3 ex_Color;
  in  vec3 ex_Normals;
  in float ex_Pz;

  out vec4 out_Color;

  vec3 test;
  float zBuffer;

  void main(void) {
    test.x = gl_PointCoord.x - 0.5;
    test.y = gl_PointCoord.y - 0.5;
    test.z = -(ex_Normals.x/ex_Normals.z) * test.x - (ex_Normals.y/ex_Normals.z) * test.y;
    if (length(test) > 0.5)
      discard;

    zBuffer = (ex_Pz + test.z * 0.1);
    gl_FragDepth = ((1.0 / zBuffer) * ( (f * n) / (f - n) ) + ( f / (f - n) ));

    out_Color = vec4(ex_Color,1.0);
  }
\end{lstlisting}
\vfill

\section{Perspective Correct Rasterization \label{perspective}}

\textbf{Vertex Shader}
\begin{lstlisting} [frame=single]
  #version 400
  uniform mat4 viewMatrix, projMatrix;
  uniform mat3 normalMatrix;
  uniform int h; //Height of the viewport
  uniform float n; //Near parameter of the viewing frustum
  uniform float t; //Top parameter of the viewing frustum
  uniform float b; //Bottom parameter of the viewing frustum
  uniform float userRadiusFactor; //Splat's radii
  uniform bool automaticRadiusEnabled;

  in float in_Radius;
  in  vec3 in_Position;
  in  vec3 in_Color;
  in  vec3 in_Normals;

  out vec3 ex_Color;
  out float ex_Radius;

  out vec4 ccPosition; //position in Camera Coordinates
  out vec3 normals;

  void main(void) {
    normals = normalize(normalMatrix * in_Normals);

    ex_Radius = in_Radius;

    ccPosition = viewMatrix * vec4(in_Position, 1.0);
    gl_Position = projMatrix * ccPosition;
    gl_PointSize = 2 * ex_Radius * (n / ccPosition.z) * (h / (t-b));

    //BackFace Culling
    if (dot (ccPosition.xyz, normals) > 0)
      gl_Position.w = 0;

    ex_Color = in_Color;
  }
\end{lstlisting}
\newpage
\textbf{Fragment Shader}
\begin{lstlisting} [frame=single]
  #version 400
  uniform mat4 viewMatrix;
  uniform float n; //Near parameter of the viewing frustum
  uniform float f; //Far parameter of the viewing frustum
  uniform float t; //Top parameter of the viewing frustum
  uniform float b; //Bottom parameter of the viewing frustum
  uniform float r; //Right parameter of the viewing frustum
  uniform float l; //Left parameter of the viewing frustum
  uniform int h;   //Height of the viewport
  uniform int w;   //Width of the viewport
  uniform int lightCount;
  uniform vec3 lightPosition[16];
  uniform vec3 lightColor[16];
  uniform float lightIntensity[16];

  in float ex_Radius;
  in  vec3 ex_Color;
  in  vec3 normals;
  in  vec4 ccPosition;

  out vec4 out_Color;

  void main(void) {
    vec3 qn;
    qn.x = (gl_FragCoord.x ) *  ((r - l)/w ) - ( (r - l)/2.0 );
    qn.y = (gl_FragCoord.y ) *  ((b - t)/h ) - ( (b - t)/2.0 );
    qn.z = -n;

    float denom = dot (qn, normals);
    if (denom == 0.0)
      discard;

    float timef = dot (ccPosition.xyz, normals ) / denom;
    vec3 q = qn * timef;
    vec3 testq = q;
    vec3 dist = (q - ccPosition.xyz);

    if ((dist.x * dist.x) + (dist.y * dist.y) + (dist.z * dist.z) > pow(ex_Radius, 2))
      discard;

    gl_FragDepth = ((1.0 / q.z) * ( (f * n) / (f - n) ) + ( f / (f - n) ));

    out_Color = vec4(ex_Color, 1.0f);
  }
\end{lstlisting}


\section{Gouraud \label{gouraud}}
\subsection{Visibility Pass}
\textbf{Vertex Shader}
\begin{lstlisting} [frame=single]
  #version 410
  uniform mat4 viewMatrix, projMatrix;
  uniform mat3 normalMatrix;
  uniform int h; //Height of the viewport
  uniform float n; //Near parameter of the viewing frustum
  uniform float t; //Top parameter of the viewing frustum
  uniform float b; //Bottom parameter of the viewing frustum

  in  vec3 in_Position;
  in  vec3 in_Normals;
  in  float in_Radius;

  out float ex_Radius;

  out vec4 ccPosition; //position in Camera Coordinates
  out vec3 normals;


  void main(void) {
    ex_Radius = in_Radius;

    normals = normalize(normalMatrix * in_Normals);

    ccPosition = viewMatrix * vec4(in_Position, 1.0);
    gl_Position = projMatrix * ccPosition;
    gl_PointSize = 2 * ex_Radius * (n / ccPosition.z) * (h / (t-b));
  }
\end{lstlisting}
\newpage

\textbf{Fragment Shader}
\begin{lstlisting} [frame=single]
  #version 410
  uniform float n; //Near parameter of the viewing frustum
  uniform float f; //Far parameter of the viewing frustum
  uniform float t; //Top parameter of the viewing frustum
  uniform float b; //Bottom parameter of the viewing frustum
  uniform float r; //Right parameter of the viewing frustum
  uniform float l; //Left parameter of the viewing frustum
  uniform int h;   //Height of the viewport
  uniform int w;   //Width of the viewport

  in float ex_Radius;
  in  vec3 normals;
  in  vec4 ccPosition;

  void main(void) {
    vec3 qn;
    qn.x = (gl_FragCoord.x ) *  ((r - l)/w ) - ( (r - l)/2.0 );
    qn.y = (gl_FragCoord.y ) *  ((b - t)/h ) - ( (b - t)/2.0 );
    qn.z = -n;

    float denom = dot (qn, normals);

    if (denom == 0.0)
      discard;

    float timef = dot (ccPosition.xyz, normals ) / denom;

    vec3 q = qn * timef;

    vec3 dist = (q - ccPosition.xyz);

    if ((dist.x * dist.x) + (dist.y * dist.y) + (dist.z * dist.z) > pow(ex_Radius, 2))
      discard;

    gl_FragDepth = ((1.0 / q.z) * ( (f * n) / (f - n) ) + ( f / (f - n) ));
  }
\end{lstlisting}
\newpage

\subsection{Blending Pass}
\textbf{Vertex Shader}
\begin{lstlisting} [frame=single]
  #version 410
  uniform mat4 viewMatrix, projMatrix;
  uniform mat3 normalMatrix;
  uniform int h; //Height of the viewport
  uniform float n; //Near parameter of the viewing frustum
  uniform float t; //Top parameter of the viewing frustum
  uniform float b; //Bottom parameter of the viewing frustum

  in float in_Radius;
  in  vec3 in_Position;
  in  vec3 in_Color;
  in  vec3 in_Normals;

  out vec3 ex_Color;
  out float ex_Radius;

  out vec4 ccPosition; //position in Camera Coordinates
  out vec3 normals;

  void main(void) {
    normals = normalize(normalMatrix * in_Normals);

    ex_Radius = in_Radius;

    ccPosition = viewMatrix * vec4(in_Position, 1.0);
    gl_Position = projMatrix * ccPosition;
    gl_PointSize = 2*ex_Radius * (n / ccPosition.z) * (h / (t-b));

    //Backface Culling
    if (dot (ccPosition.xyz, normals) > 0)
      gl_Position.w = 0;

    ex_Color = in_Color;
  }
\end{lstlisting}
\newpage

\textbf{Fragment Shader}
\begin{lstlisting} [frame=single]
  #version 410
  uniform mat4 viewMatrix;
  uniform float n; //Near parameter of the viewing frustum
  uniform float f; //Far parameter of the viewing frustum
  uniform float t; //Top parameter of the viewing frustum
  uniform float b; //Bottom parameter of the viewing frustum
  uniform float r; //Right parameter of the viewing frustum
  uniform float l; //Left parameter of the viewing frustum
  uniform int h;   //Height of the viewport
  uniform int w;   //Width of the viewport

  uniform int lightCount;
  uniform vec3 lightPosition[16];
  uniform vec3 lightColor[16];
  uniform float lightIntensity[16];

  in float ex_Radius;
  in  vec3 ex_Color;
  in  vec3 normals;
  in  vec4 ccPosition;

  layout (location = 0) out vec4 out_Color;

  void main(void) {
    vec3 qn;
    qn.x = (gl_FragCoord.x ) *  ((r - l)/w ) - ( (r - l)/2.0 );
    qn.y = (gl_FragCoord.y ) *  ((b - t)/h ) - ( (b - t)/2.0 );
    qn.z = -n;

    float denom = dot (qn, normals);

    if (denom == 0.0)
      discard;

    float timef = dot (ccPosition.xyz, normals ) / denom;

    vec3 q = qn * timef;
    vec3 testq = q;

    vec3 dist = (q - ccPosition.xyz);

    if ((dist.x * dist.x) + (dist.y * dist.y) + (dist.z * dist.z) > pow(ex_Radius, 2))
      discard;

    vec3 epsilon = normalize(qn)/40.0f;
    q = q - epsilon;

    gl_FragDepth = ((1.0 / q.z) * ( (f * n) / (f - n) ) + ( f / (f - n) ));
    float weight = (1.0f - length(dist)/ex_Radius);

    vec3 color = ex_Color;

    //Diffuse
    vec3 dotValue = vec3(0,0,0);
    for (int i = 0; i < lightCount; i++) {
      vec3 ccLightPosition = (viewMatrix * vec4(lightPosition[i], 1.0f)).xyz;
      vec3 lithToQ = normalize(ccLightPosition - testq);
      dotValue += vec3(max(dot(normals, lithToQ), 0.0)) * lightIntensity[i] * lightColor[i];
    }

    out_Color = vec4(dotValue + color, 1.0f);
  }
\end{lstlisting}
\newpage

\subsection{Normalization Pass}
\textbf{Vertex Shader}
\begin{lstlisting} [frame=single]
  #version 410
  in  vec3 in_Position;
  in  vec3 in_Color;

  out vec4 out_Color;

  void main(void) {
    gl_Position = vec4(in_Position, 1.0);
  }
\end{lstlisting}

\textbf{Fragment Shader}
\begin{lstlisting} [frame=single]
  #version 410
  uniform sampler2DRect blendTexture;

  out vec4 out_Color;

  void main(void) {
    vec4 textureColor = texture(blendTexture, gl_FragCoord.xy);

    if (textureColor.a <= 0.0f)
      discard;

    out_Color = vec4(textureColor.rgb/textureColor.a, 1.0f);
  }
\end{lstlisting}
\newpage

\section{Phong \label{phong}}
\subsection{Visibility Pass}
\textbf{Vertex Shader}
\begin{lstlisting} [frame=single]
  #version 410
  uniform mat4 viewMatrix, projMatrix;
  uniform mat3 normalMatrix;
  uniform int h; //Height of the viewport
  uniform float n; //Near parameter of the viewing frustum
  uniform float t; //Top parameter of the viewing frustum
  uniform float b; //Bottom parameter of the viewing frustum

  in  vec3 in_Position;
  in  vec3 in_Normals;
  in  float in_Radius;

  out float ex_Radius;

  out vec4 ccPosition; //position in Camera Coordinates
  out vec3 normals;

  void main(void) {
    ex_Radius = in_Radius;

    normals = normalize(normalMatrix * in_Normals);

    ccPosition = viewMatrix * vec4(in_Position, 1.0);
    gl_Position = projMatrix * ccPosition;
    gl_PointSize = 2 * ex_Radius * (n / ccPosition.z) * (h / (t-b));
  }
\end{lstlisting}
\newpage

\textbf{Fragment Shader}
\begin{lstlisting} [frame=single]
  #version 410
  uniform float n; //Near parameter of the viewing frustum
  uniform float f; //Far parameter of the viewing frustum
  uniform float t; //Top parameter of the viewing frustum
  uniform float b; //Bottom parameter of the viewing frustum
  uniform float r; //Right parameter of the viewing frustum
  uniform float l; //Left parameter of the viewing frustum
  uniform int h;   //Height of the viewport
  uniform int w;   //Width of the viewport

  in float ex_Radius;
  in  vec3 normals;
  in  vec4 ccPosition;

  void main(void) {
    vec3 qn;
    qn.x = (gl_FragCoord.x ) *  ((r - l)/w ) - ( (r - l)/2.0 );
    qn.y = (gl_FragCoord.y ) *  ((b - t)/h ) - ( (b - t)/2.0 );
    qn.z = -n;

    float denom = dot (qn, normals);

    if (denom == 0.0)
      discard;

    float timef = dot (ccPosition.xyz, normals ) / denom;

    vec3 q = qn * timef;

    vec3 dist = (q - ccPosition.xyz);

    if ((dist.x * dist.x) + (dist.y * dist.y) + (dist.z * dist.z) > pow(ex_Radius, 2))
      discard;

    gl_FragDepth = ((1.0 / q.z) * ( (f * n) / (f - n) ) + ( f / (f - n) ));
  }
\end{lstlisting}
\newpage

\subsection{Blending Pass}
\textbf{Vertex Shader}
\begin{lstlisting} [frame=single]
  #version 400
  uniform mat4 viewMatrix, projMatrix;
  uniform mat3 normalMatrix;
  uniform int h; //Height of the viewport
  uniform float n; //Near parameter of the viewing frustum
  uniform float t; //Top parameter of the viewing frustum
  uniform float b; //Bottom parameter of the viewing frustum

  in float in_Radius;
  in  vec3 in_Position;
  in  vec3 in_Color;
  in  vec3 in_Normals;

  out vec3 ex_Color;
  out float ex_Radius;

  out vec4 ccPosition; //position in Camera Coordinates
  out vec3 normals;

  void main(void) {
    normals = normalize(normalMatrix * in_Normals);

    ex_Radius = in_Radius;

    //p. 277
    ccPosition = viewMatrix * vec4(in_Position, 1.0);
    gl_Position = projMatrix * ccPosition;
    gl_PointSize = 2 * ex_Radius * (n / ccPosition.z) * (h / (t-b));

    //BackFace Culling
    if (dot (ccPosition.xyz, normals) > 0)
      gl_Position.w = 0;

    ex_Color = in_Color;
  }
\end{lstlisting}
\newpage

\textbf{Fragment Shader}
\begin{lstlisting} [frame=single]
  #version 400
  uniform mat4 viewMatrix;
  uniform float n; //Near parameter of the viewing frustum
  uniform float f; //Far parameter of the viewing frustum
  uniform float t; //Top parameter of the viewing frustum
  uniform float b; //Bottom parameter of the viewing frustum
  uniform float r; //Right parameter of the viewing frustum
  uniform float l; //Left parameter of the viewing frustum
  uniform int h;   //Height of the viewport
  uniform int w;   //Width of the viewport
  uniform bool colorEnabled;
  uniform int lightCount;
  uniform vec3 lightPosition[16];
  uniform vec3 lightColor[16];
  uniform float lightIntensity[16];

  in float ex_Radius;
  in  vec3 ex_Color;
  in  vec3 normals;
  in  vec4 ccPosition;

  out vec4 out_Color;

  void main(void) {
    vec3 qn;
    qn.x = (gl_FragCoord.x ) *  ((r - l)/w ) - ( (r - l)/2.0 );
    qn.y = (gl_FragCoord.y ) *  ((b - t)/h ) - ( (b - t)/2.0 );
    qn.z = -n;

    float denom = dot (qn, normals);

    if (denom == 0.0)
      discard;

    float timef = dot (ccPosition.xyz, normals ) / denom;

    vec3 q = qn * timef;
    vec3 testq = q;

    vec3 dist = (q - ccPosition.xyz);

    if ((dist.x * dist.x) + (dist.y * dist.y) + (dist.z * dist.z) > pow(ex_Radius, 2))
      discard;

    vec3 epsilon = normalize(qn)/40.0f;
    q = q - epsilon;

    gl_FragDepth = ((1.0 / q.z) * ( (f * n) / (f - n) ) + ( f / (f - n) ));

    //Phong
    float weight = (1.0f - length(dist)/ex_Radius);
    vec3 phongNormal = normalize((q + normals) - ccPosition.xyz);

    vec3 color = ex_Color;
    //Diffuse
    vec3 dotValue = vec3(0,0,0);
    for (int i = 0; i < lightCount; i++) {
      vec3 ccLightPosition = (viewMatrix * vec4(lightPosition[i], 1.0f)).xyz;
      vec3 lithToQ = normalize(ccLightPosition - testq);
      dotValue += vec3(max(dot(normals, lithToQ), 0.0)) * lightIntensity[i] * lightColor[i];
    }

    out_Color = vec4(dotValue + color, 1.0f * weight);
  }
\end{lstlisting}
\newpage

\subsection{Normalization Pass}
\textbf{Vertex Shader}
\begin{lstlisting} [frame=single]
  #version 410
  in  vec3 in_Position;
  in  vec3 in_Color;

  out vec4 out_Color;

  void main(void) {
    gl_Position = vec4(in_Position, 1.0);
  }
\end{lstlisting}

\textbf{Fragment Shader}
\begin{lstlisting} [frame=single]
  #version 410
  uniform sampler2DRect blendTexture;

  out vec4 out_Color;

  void main(void) {
    vec4 textureColor = texture(blendTexture, gl_FragCoord.xy);

    if (textureColor.a <= 0.0f)
      discard;

    out_Color = vec4(textureColor.rgb/textureColor.a, 1.0f);
  }
\end{lstlisting}
\newpage

\section{Deferred \label{deferred}}
\subsection{Visibility Pass}
\textbf{Vertex Shader}
\begin{lstlisting} [frame=single]
  #version 410
  uniform mat4 viewMatrix, projMatrix;
  uniform mat3 normalMatrix;
  uniform int h; //Height of the viewport
  uniform float n; //Near parameter of the viewing frustum
  uniform float t; //Top parameter of the viewing frustum
  uniform float b; //Bottom parameter of the viewing frustum

  in  vec3 in_Position;
  in  vec3 in_Normals;
  in  float in_Radius;

  out float ex_Radius;

  out vec4 ccPosition; //position in Camera Coordinates
  out vec3 normals;

  void main(void) {
    ex_Radius = in_Radius;

    normals = normalize(normalMatrix * in_Normals);

    ccPosition = viewMatrix * vec4(in_Position, 1.0);
    gl_Position = projMatrix * ccPosition;
    gl_PointSize = 2 * ex_Radius * (n / ccPosition.z) * (h / (t-b));
  }
\end{lstlisting}
\newpage

\textbf{Fragment Shader}
\begin{lstlisting} [frame=single]
  #version 410
  uniform float n; //Near parameter of the viewing frustum
  uniform float f; //Far parameter of the viewing frustum
  uniform float t; //Top parameter of the viewing frustum
  uniform float b; //Bottom parameter of the viewing frustum
  uniform float r; //Right parameter of the viewing frustum
  uniform float l; //Left parameter of the viewing frustum
  uniform int h;   //Height of the viewport
  uniform int w;   //Width of the viewport

  in float ex_Radius;
  in  vec3 normals;
  in  vec4 ccPosition;

  layout (location = 1) out vec3 out_Position;

  void main(void) {
    vec3 qn;
    qn.x = (gl_FragCoord.x ) *  ((r - l)/w ) - ( (r - l)/2.0 );
    qn.y = (gl_FragCoord.y ) *  ((b - t)/h ) - ( (b - t)/2.0 );
    qn.z = -n;

    float denom = dot (qn, normals);

    if (denom == 0.0)
      discard;

    float timef = dot (ccPosition.xyz, normals ) / denom;

    vec3 q = qn * timef;

    vec3 dist = (q - ccPosition.xyz);

    if ((dist.x * dist.x) + (dist.y * dist.y) + (dist.z * dist.z) > pow(ex_Radius, 2))
      discard;

    out_Position = q;
    gl_FragDepth = ((1.0 / q.z) * ( (f * n) / (f - n) ) + ( f / (f - n) ));
  }
\end{lstlisting}
\newpage

\subsection{Blending Pass}
\textbf{Vertex Shader}
\begin{lstlisting} [frame=single]
  #version 410
  uniform mat4 viewMatrix, projMatrix;
  uniform mat3 normalMatrix;
  uniform int h; //Height of the viewport
  uniform float n; //Near parameter of the viewing frustum
  uniform float t; //Top parameter of the viewing frustum
  uniform float b; //Bottom parameter of the viewing frustum

  in float in_Radius;
  in  vec3 in_Position;
  in  vec3 in_Color;
  in  vec3 in_Normals;

  out vec3 ex_Color;
  out float ex_Radius;

  out vec4 ccPosition; //position in Camera Coordinates
  out vec3 normals;

  void main(void) {
    normals = normalize(normalMatrix * in_Normals);

    ex_Radius = in_Radius;

    ccPosition = viewMatrix * vec4(in_Position, 1.0);
    gl_Position = projMatrix * ccPosition;
    gl_PointSize = 2*ex_Radius * (n / ccPosition.z) * (h / (t-b));

    //Backface Culling
    if (dot (ccPosition.xyz, normals) > 0)
      gl_Position.w = 0;

    ex_Color = in_Color;
  }
\end{lstlisting}
\newpage

\textbf{Fragment Shader}
\begin{lstlisting} [frame=single]
  #version 410
  uniform float n; //Near parameter of the viewing frustum
  uniform float f; //Far parameter of the viewing frustum
  uniform float t; //Top parameter of the viewing frustum
  uniform float b; //Bottom parameter of the viewing frustum
  uniform float r; //Right parameter of the viewing frustum
  uniform float l; //Left parameter of the viewing frustum
  uniform int h;   //Height of the viewport
  uniform int w;   //Width of the viewport

  in float ex_Radius;
  in  vec3 ex_Color;
  in  vec3 normals;
  in  vec4 ccPosition;

  layout (location = 0) out vec4 out_Color;
  layout (location = 1) out vec4 out_Normals;

  void main(void) {
    vec3 qn;
    qn.x = (gl_FragCoord.x ) *  ((r - l)/w ) - ( (r - l)/2.0 );
    qn.y = (gl_FragCoord.y ) *  ((b - t)/h ) - ( (b - t)/2.0 );
    qn.z = -n;

    float denom = dot (qn, normals);
    if (denom == 0.0)
      discard;

    float timef = dot (ccPosition.xyz, normals ) / denom;
    vec3 q = qn * timef;
    vec3 dist = (q - ccPosition.xyz);

    if ((dist.x * dist.x) + (dist.y * dist.y) + (dist.z * dist.z) > pow(ex_Radius, 2))
      discard;

    vec3 epsilon = normalize(qn)/40.0f;
    q = q - epsilon;

    gl_FragDepth = ((1.0 / q.z) * ( (f * n) / (f - n) ) + ( f / (f - n) ));
    float weight = (1.0f - length(dist)/ex_Radius);

    out_Color = vec4(ex_Color.rgb, 1.0f * weight); 
    out_Normals = vec4(normals, 1.0f * weight);
  }
\end{lstlisting}
\newpage

\subsection{Normalization Pass}
\textbf{Vertex Shader}
\begin{lstlisting} [frame=single]
  #version 400
  in  vec3 in_Position;
  in  vec3 in_Color;

  out vec4 out_Color;

  void main(void) {
    gl_Position = vec4(in_Position, 1.0);
  }
\end{lstlisting}

\textbf{Fragment Shader}
\begin{lstlisting} [frame=single]
  #version 410
  uniform mat4 viewMatrix;
  uniform float n; //Near parameter of the viewing frustum
  uniform float f; //Far parameter of the viewing frustum
  uniform float t; //Top parameter of the viewing frustum
  uniform float b; //Bottom parameter of the viewing frustum
  uniform float r; //Right parameter of the viewing frustum
  uniform float l; //Left parameter of the viewing frustum
  uniform int h;   //Height of the viewport
  uniform int w;   //Width of the viewport

  uniform int lightCount;
  uniform vec3 lightPosition[16];
  uniform vec3 lightColor[16];
  uniform float lightIntensity[16];

  uniform sampler2DRect blendTexture;
  uniform sampler2DRect normalTexture;
  uniform sampler2DRect positionTexture;

  out vec4 out_Color;

  void main(void) {
    vec4 textureColor = texture(blendTexture, gl_FragCoord.xy);

    if (textureColor.a <= 0.0f)
      discard;

    vec4 textureNormal = texture(normalTexture, gl_FragCoord.xy);

    vec4 normalizedColor = vec4(textureColor.rgb/textureColor.a, 1.0f);
    vec4 normalizedNormal = vec4(textureNormal.xyz/textureNormal.w, 1.0f);
    vec3 q = texture(positionTexture, gl_FragCoord.xy).xyz;

    vec3 color = normalizedColor.rgb;

    //Lightning with the resultant normalized textures
    vec3 dotValue = vec3(0,0,0);
    for (int i = 0; i < lightCount; i++) {
      vec3 ccLightPosition = (viewMatrix * vec4(lightPosition[i],1.0f)).xyz;
      vec3 ligthToQ = normalize(ccLightPosition - q);
      dotValue += vec3(max(dot(normalize(normalizedNormal.xyz), ligthToQ), 0.0)) * lightIntensity[i] * lightColor[i];
    }

    out_Color = vec4(dotValue + color, 1.0f);
  }
\end{lstlisting}
